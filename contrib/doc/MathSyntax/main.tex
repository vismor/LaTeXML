\documentclass[a4paper,12pt]{memoir}
\usepackage{graphicx}
\usepackage[usenames,svgnames]{xcolor}
\chapterstyle{bianchi}
\usepackage{a4wide,amsfonts,appendix}
\usepackage[show]{ed}
\usepackage{url}
\usepackage[colorlinks=true,linkcolor=blue,urlcolor=blue]{hyperref}
\usepackage{xspace}

% Less detailed TOC
\setcounter{tocdepth}{1}
\def\arxiv{\textsc{arXiv}\xspace}

\begin{document}
%% Some shades
\definecolor{Dark}{gray}{0.2}
\definecolor{MedDark}{gray}{0.4}
\definecolor{Medium}{gray}{0.6}
\definecolor{Light}{gray}{0.8}
\newlength{\drop}% for my convenience

\renewcommand{\maketitle}{\begingroup%
\drop = 0.1\textheight
\fboxsep 0.5\baselineskip
\sffamily
\vspace*{\drop}
\centering
{\textcolor{NavyBlue}{\HUGE The Structure of Mathematical Expressions}}\par
\vspace{0.5\drop}
\colorbox{Dark}{\textcolor{white}{\normalfont\itshape\Large
An {\textsc{arXiv}} Case Study}}\par
\vspace{\drop}
{\Large Deyan Ginev and Bruce R. Miller}\par
\vspace{0.05\textheight}
{\large National Institute of Standards and Techonology}\par
\vfill
{\begin{center}\today\end{center}}
\vspace*{\drop}
{\hfill\includegraphics{NIST}}
\endgroup}
\makeatother

\pagestyle{empty}
\maketitle
\newpage
\pagestyle{plain}
% Remove parskip for toc
\setlength{\parskip}{0ex plus 0.5ex minus 0.2ex}
\tableofcontents

\chapter{Introduction}
In this study, we survey the notational diversity of present-day mathematical expressions, in order to uncover their linguistic phenomena. A practical motivation for this study is to provide a foundation for determining the boundary between syntactic and semantic phenomena in said expressions, from the perspective of language modeling. The ultimate goal of this project is to construct a grammar of mathematical expressions, which captures all relevant syntactic properties established in this study, and allows for the semantic analysis necessary to model and observe the semantic relationships.

\section{Motivation}
We want to enable machine-reading of formulas, in order to provide a variety of user-assistance services, such as semantic search, text-to-speech synthesis, semantic interactions (definition lookup), as well as computer algebra support (``evaluate subexpressions on demand'') and ultimately computer verification (``does that proof step really hold?'').\ednote{expand}
\section{Related Resources}
Notation census, beginnings of study are in Deyan's thesis, Naproche and FMathL have examples, but no real systematic study.\ednote{expand}

\section{Experimental Setup}
The primary corpus on which we base this investigation is the Cornel pre-print archive ``\arxiv''\ednote{cite here}, consisting of over 700,000 articles in 37 scientific subfields.
\subsection{\arxiv Sandbox}
\ednote{Say that, on the \arxiv front, we first start with the train sandbox from Deyan's thesis}
\begin{table}\begin{center}
\begin{tabular}{|ll|}\hline
Train1 & Differential Geometry \\ & \url{http://arxmliv.kwarc.info/files/9609/dg-ga.9609012} \\[2mm]
Train2 & Quantum Physics \\  & \url{http://arxmliv.kwarc.info/files/0910/0910.5733/} \\[2mm]
Train3 & High Energy Physics - Theory \\  & \url{http://arxmliv.kwarc.info/files/9407/hep-th.9407125/} \\[2mm]
Train4 & Commutative Algebra \\  & \url{http://arxmliv.kwarc.info/files/0809/0809.4873/} \\[2mm]
Train5 & Statistics Theory \\  & \url{http://arxmliv.kwarc.info/files/0905/0905.1486/} \\[2mm]
Train6 & General Relativity and Quantum Cosmology \\  & \url{http://arxmliv.kwarc.info/files/0807/0807.2507/} \\[2mm]
Train7 & Cosmology and Extragalactic Astrophysics \\  & \url{http://arxmliv.kwarc.info/files/0908/0908.2548} \\[2mm]
Train8 & Exactly Solvable and Integrable Systems \\  & \url{http://arxmliv.kwarc.info/files/0905/0905.2033} \\[2mm]
Train9 & Geometric Topology \\  & \url{http://arxmliv.kwarc.info/files/0809/0809.4477} \\[2mm]
Train10 & Algebraic Geometry \\  & \url{http://arxmliv.kwarc.info/files/0704/0704.0537} \\ \hline
\end{tabular}
\caption{Sandbox of Ten Random \arxiv Papers from Diverse Scientific Subfields}
\end{center}
\end{table}

As a secondary resource, we we will also consult entry-level literature on highschool mathematics, in order to exhibit basic phenomena, as well as to demonstrate phenomena apriori known to the authors.\ednote{Wikipedia? PEMDAS?}


\chapter{A Study of Mathematical Syntax}

\section{Basics}
\input{basics}
\section{Discrete math}
\subsection{Set Theoretic Notations}
\ednote{elementhood, inclusions, set constructors, overloaded arith ops}
\ednote{also maps : domains -> codomains, xRy notations}
\input{logic}
\input{combinatorics}
\input{numbertheory}
\input{graphs}
\input{algebra}
\input{functions}

\section{Continuous math}
\input{calculus}
\input{probability}
\input{intervals}
\input{topology}

\section{Other fields}
\input{quantum}
\ednote{computer science, biology, chemistry...}
$\vdots$

\chapter{Discussion}

\chapter{Conclusion}

\end{document}