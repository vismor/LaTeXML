\documentclass[a4paper,12pt]{memoir}
\usepackage{graphicx}
\usepackage[usenames,svgnames]{xcolor}
\chapterstyle{bianchi}
\usepackage{a4wide,amsfonts,appendix}
\usepackage[show]{ed}
\usepackage{url}

% Less detailed TOC
\setcounter{tocdepth}{1}


\begin{document}
%% Some shades
\definecolor{Dark}{gray}{0.2}
\definecolor{MedDark}{gray}{0.4}
\definecolor{Medium}{gray}{0.6}
\definecolor{Light}{gray}{0.8}
\newlength{\drop}% for my convenience

\renewcommand{\maketitle}{\begingroup%
\drop = 0.1\textheight
\fboxsep 0.5\baselineskip
\sffamily
\vspace*{\drop}
\centering
{\textcolor{NavyBlue}{\HUGE The Structure of Mathematical Expressions}}\par
\vspace{0.5\drop}
\colorbox{Dark}{\textcolor{white}{\normalfont\itshape\Large
An {\textsc{arXiv}} Case Study}}\par
\vspace{\drop}
{\Large Deyan Ginev and Bruce R. Miller}\par
\vspace{0.05\textheight}
{\large National Institute of Standards and Techonology}\par
\vfill
{\begin{center}\today\end{center}}
\vspace*{\drop}
{\hfill\includegraphics{NIST}}
\endgroup}
\makeatother

\pagestyle{empty}
\maketitle
\newpage
\pagestyle{plain}
% Remove parskip for toc
\setlength{\parskip}{0ex plus 0.5ex minus 0.2ex}
\tableofcontents

\chapter{Introduction}
In this study, we survey the notational diversity of present-day mathematical expressions, in order to uncover its linguistic phenomena.
\section{Motivation}
We want to enable machine-reading of formulas, in order to provide a variety of user-assistance services, such as semantic search, text-to-speech synthesis, semantic interactions (definition lookup), as well as computer algebra support (``evaluate subexpressions on demand'') and ultimately computer verification (``does that proof step really hold?'').\ednote{expand}
\section{Related Resources}
Notation census, beginnings of study are in Deyan's thesis, Naproche and FMathL have examples, but no real systematic study.\ednote{expand}

\section{Experimental Setup}
The primary corpus on which we base this investigation is the Cornel pre-print archive ``arXiv''\ednote{cite here}, consisting of over 700,000 articles in 37 scientific subfields.

As a secondary resource, we we will also consult entry-level literature on highschool mathematics, in order to exhibit basic phenomena, as well as to demonstrate phenomena apriori known to the authors

\chapter{A Study of Mathematical Syntax}

\section{Basics}
\input{basics}
\section{Discrete math}
\subsection{Set Theoretic Notations}
\ednote{elementhood, inclusions, set constructors, overloaded arith ops}
\ednote{also maps : domains -> codomains, xRy notations}
\input{logic}
\input{combinatorics}
\input{numbertheory}
\input{graphs}
\input{algebra}
\input{functions}

\section{Continuous math}
\input{calculus}
\input{probability}
\input{intervals}
\input{topology}

\section{Other fields}
\input{quantum}
\ednote{computer science, biology, chemistry...}
$\vdots$

\chapter{Discussion}

\chapter{Conclusion}

\end{document}