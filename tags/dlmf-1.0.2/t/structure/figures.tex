\documentclass{article}
\usepackage{subfigure}
\begin{document}
\paragraph{Figures}
Two figures, \ref{fig:A-one} and \ref{fig:A-two}.

\begin{figure}[ht]
\begin{center}
 [Graphic One]\\
\caption{This is a figure\label{fig:A-one}}
\end{center}
\end{figure}

\begin{figure}[ht]
\begin{center}
[Graphic Two]\\
\caption{This is another figure\label{fig:A-two}}
\end{center}
\end{figure}

\paragraph{Sub-Figures with captions}
Two figures, \ref{fig:B-one} and \ref{fig:B-two}.

\begin{figure}[ht]
\begin{center}
\subfigure[This is a figure\label{fig:B-one}]{[Graphic One]}
\subfigure[This is another figure\label{fig:B-two}]{[Graphic Two]}\\
\end{center}
\end{figure}

\paragraph{Sub-Figures, collective caption}
Two figures, \ref{fig:C}

\begin{figure}[ht]
\begin{center}
\subfigure{[Graphic One]}
\subfigure{[Graphic Two]}\\
\end{center}
\caption{Collectively two figures\label{fig:C}}
\end{figure}

\paragraph{Sub-Figures with all captions}
Two figures, \ref{fig:D-one} and \ref{fig:D-two},
and collectively as \ref{fig:D}.

\begin{figure}[ht]
\begin{center}
\subfigure[This is a figure\label{fig:D-one}]{[Graphic One]}
\subfigure[This is another figure\label{fig:D-two}]{[Graphic Two]}\\
\caption{Collectively two figures\label{fig:D}}
\end{center}
\end{figure}

\end{document}